\begin{problem}{\kcpcansubinsutitle}
    {표준 입력}{표준 출력}
    {\kcpcansubinsutime\,초}{\kcpcansubinsumemory\,MB}{}{\kcpcansubinsuscore}
    
    자릿수의 합이란 무엇인가? 이는 수를 10진수로 나타내었을 때, 각 자리 숫자들의 합을 의미한다. 예를 들어, 1093의 자릿수의 합은 1 + 0 + 9 + 3 = 13 이다.
    
    우리는 자릿수의 합이 짝수인 양의 정수를 수빈수라고 부르기로 했다. 그리고 수빈수가 아닌 양의 정수를 안수빈수라고 부르기로 했다.
    
    어떤 양의 정수 $ N $이 주어졌을 때, $ N $의 배수 중 안수빈수가 있는지 확인하고, 있다면 아무거나 하나를 출력하는 프로그램을 작성하시오.
    
    \InputFile
    첫 번째 줄에 테스트케이스의 개수 $ T $가 주어진다. $ (1 \leq T \leq 1,000) $
    
    두 번째 줄부터 $ T $개의 줄에 걸쳐, 각각의 테스트 케이스에 대한 $ N $이 주어진다. $ (1 \leq N \leq 100,000,000) $
    
    
    \OutputFile
    각 테스트 케이스에 대해 매 줄마다 아래와 같이 출력한다.
    \begin{itemize}
        \item $ N $의 배수 중 안수빈수가 없다면 \verb|-1|을 출력한다.
        \item $ N $의 배수 중 안수빈수가 있다면, 그 중 $ 10^{18} $ 이하의 안수빈수를 아무거나 하나 출력한다.
    \end{itemize}
    100,000,000 이하의 양의 정수 $ N $에 대해, $ N $의 배수 중 안수빈수가 있다면, 그 중 $ 10^{18} $ 이하의 안수빈수가 존재함이 보장된다고 생각해도 좋다.
   
    \Examples
    
    \begin{example}
        \exmp{
            4
            1000
            1234
            13
            9
        }{%
            1000
            7404
            52
            18
        }%
    \end{example}
    
\end{problem}

