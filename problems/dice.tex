\begin{problem}{\kcpcdicetitle}
    {표준 입력}{표준 출력}
    {\kcpcdicetime\,초}{\kcpcdicememory\,MB}{}{\kcpcdicescore}
    
    전설에 따르면 주사위야말로 진정한 실력을 판가름할 수 있는 도구라고 한다. 그중 1부터 6까지의 수가 적혀있는 정육면체 주사위가 그 중에서도 으뜸이라고 전해진다. 정육면체 주사위를 던졌을 때 각 면이 나올 확률은 1/6로 같다. 상헌이는 주사위의 시험을 받게 되었다. 정육면체 주사위 $ N $개를 굴려, 나온 눈의 합이 $ K $ 이상이 되어야 한다.
    
    그러나 상헌이는 주사위 컨트롤 실력이 아직 부족하므로, 주사위 $ N $개를 굴린 다음 마음에 안 드는 눈을 가진 주사위들을 선택해서 다시 한 번 굴릴 수 있는 기회를 얻었다. 여기서 어느 주사위도 굴리지 않아도 괜찮다.
    
    상헌이는 고민이 든다. 어떤 주사위를 선택해야 모든 눈의 합이 $ K $ 이상이 될 확률이 가장 높을까? 상헌이는 실력을 판가름하기 앞서 확률을 계산해보기 위해 여러분들에게 도움을 청했다.
    \InputFile
    입력의 첫 줄에는 테스트 케이스의 개수를 의미하는 정수 $ T $가 주어진다. $ (1 \leq T \leq 1,000) $
    
    각 테스트 케이스는 두 줄로 이루어져 있으며 테스트 케이스 사이에 빈 줄은 없다.
    
    각 테스트 케이스의 첫째 줄에는 상헌이가 가지고 있는 정육면체 주사위의 개수와 눈의 합의 최소 목표치를 의미하는 두 정수 $ N $과 $ K $가 공백으로 구분되어 주어진다. $ (1 \leq N \leq 20,\ N \leq K \leq 6N) $
    
    각 테스트 케이스의 둘째 줄에는 $ N $개의 정수 $ a_1,\ a_2,\ \cdots,\ a_N $이 공백으로 구분되어 주어진다. $ a_i $ 는 $ i $번째 정육면체 주사위를 던져서 나온 눈을 의미하며, 1 이상 6 이하의 자연수이다.
    
    \OutputFile
    출력은 각 테스트 케이스별로 두 줄로 이루어진다. 그러므로 총 $ 2T $개의 줄에 걸쳐서 출력을 해야 한다. 각 테스트 케이스 별로 정육면체 주사위를 적절히 선택해서 다시 굴린 후, 눈의 합이 $ K $ 이상이 될 확률의 최댓값이 $ p $라고 하자.
    
    각 테스트 케이스마다 첫째 줄에는 정수 $ 6^N $\,\texttimes\,$ p $ 를 출력한다. 이 값은 32비트 정수에 들어가기에는 매우 클 수 있다.
    
    각 테스트 케이스 별로 둘째 줄에는 어떻게 선택해서 던져야 위의 확률 $ p $가 나오는지를 의미하는 $ N $개의 정수 $ x_1,\ x_2,\ \cdots,\ x_N $ 을 공백으로 구분하여 출력한다. $ x_i $는 $ i $번째 주사위를 던져야 하는 경우 1이며 그렇지 않을 경우 0이다. 확률이 $ p $가 되게 정육면체 주사위를 선택할 수 있는 방법이 다양하다면 그중 아무 것이나 출력해도 좋다.
    
    \SubtaskWithScore{\kcpcdicesmallscore}
    $ 1 \leq N \leq 3 $이다.
    
    \SubtaskWithScore{\kcpcdicelargescore}
    문제 입력에서 주어진 조건 외에 추가적인 조건이 없다.

    \Examples
    
    \begin{example}
        \exmp{
            3
            1 5
            3
            2 10
            6 1
            3 8
            2 5 4
        }{%
            2
            1
            18
            0 1
            216
            1 0 0
        }%
    \end{example}
    
    \Explanation
    첫 번째 테스트 케이스에서는 주사위를 다시 굴리면 눈의 합이 5 이상이 될 확률이 1/3이며 굴리지 않으면 0이다.
    
    두 번째 테스트 케이스에서는 두 번째 주사위만 굴리는 것이 최적이며 이 때의 확률은 1/2이다.
    
    세 번째 테스트 케이스에서는 이미 눈의 합이 11이다. 첫 번째 주사위를 굴린 결과와 상관없이 눈의 합이 8 이상이므로 확률은 1이다.
    
    
    
\end{problem}

