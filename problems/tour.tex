\begin{problem}{\kcpctourtitle}
    {표준 입력}{표준 출력}
    {\kcpctourtime\,초}{\kcpctourmemory\,MB}{}{\kcpctourscore}
    
    휴학을 한 창수는 열심히 알바를 해서 $ B $원을 모았다. 창수는 돈도 쌓였겠다, 홍준랜드로 여행을 가기로 결정했다.
    
    홍준랜드는 $ N $개의 여행지로 구성되어있으며, 각 여행지들은 2차원 격자판상에 존재한다. $ i $번째 여행지는 점 $ (x_i,\ y_i) $로 표현된다.
    
    홍준랜드의 특별한 규칙이 있는데, 여행지끼리 이동할 때 꼭 택시를 타야만 한다. 택시요금은 택시거리로 책정된다. 즉 $ s $번 여행지에서 $ e $번 여행지로 이동하는 데에는 $ |x_s - x_e| + |y_s - y_e| $ 만큼의 비용이 든다. 또한 홍준랜드의 각 여행지 $ (i, i+1) $ 에 대해 $ x_i \leq x_{i+1},\ y_i \leq y_{i+1} $ 이 성립한다. $ (1 \leq i < N) $ 단, 같은 점에 두 개 이상의 여행지가 있는 경우는 없다. 그리고 홍준랜드의 택시는 출발지에서 택시에 탑승하자마자 경로상의 여행지를 볼 새도 없이 도착지에 도착한다.
    
    창수는 $ B $원 이하의 비용을 써서 $ N $개의 여행지를 단 한 번씩 모두 방문하는 여행을 하고 싶어한다.
    
    문득 창수는 이런 계획이 몇 가지나 존재하는지 궁금해졌다. 창수가 세울 수 있는 여행계획의 경우의 수를 1,000,000,007로 나눈 나머지를 구하시오.
    
    
    \InputFile
    첫 번째 줄에는 홍준랜드에 있는 여행지의 개수와 창수가 희망하는 최대 비용을 의미하는 두 정수 $ N $과 $ B $가 공백으로 구분되어 주어진다. $ (1 \leq N \leq 60,\ 0 \leq B \leq 2,000) $
    
    그 이후 $ N $개의 줄에 걸쳐 여행지의 위치가 주어진다. $ i $번째 줄에는 $ i $번째 여행지의 정보인 두 정수 $ x_i,\ y_i $ 가 공백으로 구분되어 주어진다. $ (|x_i| \leq B,\ |y_i| \leq B) $
    
    % 같은 여행지는 2번 이상 방문하지 않은 건가? 아니면 N개 전부 방문?
    \OutputFile
    가능한 여행 계획의 경우의 수를 1,000,000,007로 나눈 나머지를 출력한다.

    \Examples
    
    \begin{example}
        \exmp{
            3 20
            1 1
            3 5
            9 7
        }{%
            4
        }%          
    \end{example}
    
    가능한 여행지 경로는 $ (1,2,3) $, $ (2,1,3) $, $ (3,1,2) $, $ (3,2,1) $이다.
    
\end{problem}

