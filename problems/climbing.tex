\begin{problem}{\kcpcclimbingtitle}
    {표준 입력}{표준 출력}
    {\kcpcclimbingtime\,초}{\kcpcclimbingmemory\,MB}{}{\kcpcclimbingscore}
    
    주환이는 요즘 등산에 빠졌다. 주환이는 등산을 위해 지도를 가지고 있는데, 그 지도에는 각 지점의 높이와 갈 수 있는 다른 지점까지의 거리가 표시되어 있다.
    
    주환이는 아침에 집에서 출발하여 등산을 갔다가, 오후 수업을 듣기 위해 고려대학교로 돌아와야 한다.
    
    \begin{enumerate}[label=\Alph*.]
        \item 주환이는 지도의 임의의 지점을 골라, 그 지점을 목표로 정한다. 집 또는 고려대학교는 목표로 선택할 수 없다.
        \item 주환이가 집에서 정한 목표에 도달할 때까지는 항상 높이가 증가하는 방향으로만 이동해야 한다.
        \item 주환이가 정한 목표에 도달한 후, 고려대학교로 갈 때에는 항상 높이가 감소하는 방향으로만 이동해야 한다.
        \item 주환이는 거리 1을 움직일 때 마다 $ D $의 체력이 소모된다.
        \item 주환이는 정한 목표에 도달하면 높이 1당 $ E $의 성취감을 얻는다. 즉 높이가 $ h $인 목표에 도달하면 $ hE $의 성취감을 얻는다.
    \end{enumerate}
    주환이는 이 등산의 가치를 (얻은 성취감) - (소모한 체력) 으로 계산하기로 하였다. 주환이를 위해 가치가 가장 높은 등산 경로를 선택해주자.
    
    \InputFile
    
    첫 번째 줄에 지도에 표시된 지점의 개수, 지점을 잇는 경로의 개수, 주환이의 거리 비례 체력 소모량, 높이 비례 성취감 획득량을 나타내는 정수 $ N $, $ M $, $ D $, $ E $가 공백을 사이에 두고 주어진다. $ (2 \leq N \leq 100,000,\ 1 \leq M \leq 200,000,\ 1 \leq D \leq 100,\ 1 \leq  E \leq 100) $
    
    두 번째 줄에 $ N $개의 정수 $ h_1,\ h_2,\ \cdots ,\ h_N $이 공백으로 구분되어 주어진다. $ h_i $는 $ i $번째 지점의 높이를 의미한다. $ (1 \leq h_i \leq 1,000,000,\ 1 \leq i \leq N) $
    
    세 번째 줄부터 $ M $개의 줄에 걸쳐 세 정수 $ a,\ b,\ n $이 공백으로 구분되어 주어진다. 이는 $ a $번 지점과 $ b $번 지점을 잇는 거리 $ n $의 양방향 경로가 있음을 의미한다. $ (1 \leq a,\ b \leq N,\ 1 \leq n \leq 100,000) $
    
    어떤 지점에서 다른 지점으로 가는 경로가 여러 개 있을 수도 있으며 (등산로는 여러 개가 있을 수 있다), 한 지점에서 출발해 그 지점으로 돌아가는 경로가 있을 수도 있다 (쉼터에서 몇 바퀴 돌며 쉴 수도 있다).
    
    주환이의 집은 1번 지점에 위치하고, 고려대학교는 $ N $번 지점에 위치하며 주환이의 집과 고려대학교의 높이는 1임이 보장된다.
    
    \OutputFile
    첫 번째 줄에 주환이가 얻을 수 있는 가치의 최댓값을 출력한다. 만약 조건을 만족하는 등산 경로를 선택할 수 없다면, \verb|"Impossible"|을 쌍따옴표를 제외하고 출력한다. 답이 음수일 수 있음에 유의하여라.
    
    \Examples
    
    \begin{example}
        \exmp{
            8 13 4 9
            1 4 7 3 10 2 15 1
            1 2 3
            3 4 2
            5 6 6
            7 8 2
            2 3 4
            6 7 2
            3 6 1
            4 8 3
            5 1 6
            8 3 5
            2 5 4
            4 6 3
            5 3 8
        }{%
            15
        }%
    \exmp{
            3 2 1 1
            1 1 1
            1 2 5
            2 3 5
        }{%
            Impossible
        }%
    \end{example}
    
\end{problem}

