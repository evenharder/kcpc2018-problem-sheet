\begin{problem}{\kcpcxorpokertitle}
    {표준 입력}{표준 출력}
    {\kcpcxorpokertime\,초}{\kcpcxorpokermemory\,MB}{}{\kcpcxorpokerscore}
    
    진벽이와 미야는 XOR 포커라는 게임을 하고 있다. 서로 정수가 적힌 카드를 $ N $장 받고, 받은 카드 중 일부를 적절히 골라 점수를 계산하여 점수가 더 높은 쪽이 이기는 게임이다.
    
    점수를 계산하는 방법은 다음과 같다.
    
    \begin{itemize}
        \item 주어진 카드 중 짝수 개의 카드를 고른다. (0개는 고를 수 없다)
        \item 고른 카드에 적힌 수들의 XOR 값을 점수로 한다. (여러 정수의 XOR 값의 정의는 예제 밑의 `참고 사항'에 나와 있다.)% BOJ의 경우 `노트'
    \end{itemize}
        
    예를 들어서, 미야가 현재 $ \{1, 2, 3, 3, 5\} $가 적힌 카드를 가지고 있다고 하자. $ \{2, 3, 3, 5\} $을 고르면 $ 2 \oplus 3 \oplus 3 \oplus 5 = 7$이 점수가 된다. 똑같이 7이 되는 $ \{1, 3, 5\} $는 원소의 개수가 홀수이기 때문에 고를 수 없다.
    
    미야는 승부욕이 강해서 진벽이를 꼭 이기고 싶다. 미야가 승부에서 이길 수 있도록 도와주자.
    \InputFile
    첫 번째 줄에는 미야가 가진 카드의 개수 $ N $이 주어진다. $ (2 \leq N \leq 100,000) $
    
    두 번째 줄부터 $ N $개의 줄에 걸쳐 미야의 $ i $번째 카드에 적힌 수 $ a_i $가 주어진다. $ (0 \leq a_i \leq 10^{18}) $
    
    
    \OutputFile
    첫 번째 줄에 미야가 주어진 카드 XOR 포커를 할 때 만들 수 있는 점수의 최대값을 출력한다.

    \Examples
    
    \begin{example}
        \exmp{
            5
            1
            2
            3
            3
            5
        }{%
            7
        }%
    \exmp{
            4
            8
            2
            4
            1
        }{%
            15
        }%
    \exmp{
            7
            765
            876
            961
            315
            346
            825
            283
        }{%
            1010
        }%
    \end{example}
    
    \Note
    양의 정수들 $ x_1,\ x_2,\ \cdots,\ x_N $ 의 XOR 값은 다음과 같이 정의된다.
    
    XOR 값을 $X$를 이진법으로 나타낼 때, $ x_1,\ x_2,\ \cdots,\ x_N $ 중 $ 2^k $의 자리가 1인 수가 홀수 개 있으면 $ X $의 $ 2^k $의 자리는 1이며, 짝수 개 있으면 0이다.
\end{problem}

