\begin{problem}{\kcpcwinetitle}
    {표준 입력}{표준 출력}
    {\kcpcwinetime\,초}{\kcpcwinememory\,MB}{}{\kcpcwinescore}
    
    다음은 고려대학교 포털에 나와 있는 고대 와인에 대한 글 일부이다.
    
    \begin{framed}
        개교 100주년 공식 와인 `라 까르돈느(La Cardonne) 2000년’의 뒤를 잇는 우당 박종구 회장 기부 와인 `나파밸리 2000(Napa Valley 2000)‘, 고대와인 `클라랑델 루즈 2005(Calarendelle Rouge 2005)’, `클라랑델 루즈 2009(Clarendelle Rouge 2009)', `클라랑델 루즈 2011(Clarendelle Rouge 2011)'을 판매하였으나, 재고량 소진으로 인해 ㈜고대미래 크림슨스토어에서 독점 수입하여 선보이는 `라스토 랑데올 2014(Rasteau L’Andeol 2014)’ 와 ‘샤토 클락 2014(Chateau Clarke 2014)’ 로 변경하여 판매합니다.   
    \end{framed}
    
    이 글을 본 수빈이는 고려대학교 와인 콜렉팅이라는 목표를 가지게 되었다. 하지만 그냥 모으면 재미가 없으니 규칙을 가지고 와인을 수집하려고 한다.
    
    와인을 사기로 마음먹은 해를 0년 차라고 정의하고, $ n $년 차에는 $ Kn+Pn^2 $ 만큼의 와인을 사는 것을 목표로 했다. $ K $는 수빈이의 고려대 애착 정도를 나타내는 상수이고, $ P $는 수빈이의 구매중독 정도를 나타내는 상수이다. 
    
    그렇게 수빈이는 $ C $년 동안 열심히 와인을 모았다. (마지막 해에 산 와인의 수는 $ KC + PC^2 $가 된다.)
    
    수빈이는 와인이 이제 방에 가득 쌓여, 자기가 얼마나 수집했는지 수를 세기 어려웠다.
    
    여러분이 수빈이가 와인을 얼마나 수집했는지 계산해주자.
    \InputFile
    첫 번째 줄에 수빈이가 와인은 모은 년수, 수빈이의 고려대 애착 정도, 수빈이의 구매중독 정도를 의미하는 정수 $ C $, $ K $, $ P $가 공백으로 구분되어 주어진다. $ (0 \leq C \leq 100,\ 0 \leq K \leq 1,000,\ 0 \leq P \leq 100) $
    
    \OutputFile
    첫 번째 줄에 수빈이가 C년 동안 수집한 와인 수를 출력한다.
    
    \Examples
    
    \begin{example}
        \exmp{
            3 1 1
        }{%
            20
        }%
        \exmp{
            5 28 27
        }{%
            1905
        }%
    \end{example}
    
    \Explanation
    첫 번째 예제의 경우 1년차 2병, 2년차 6병, 3년차 12병으로 총 20병을 수집하였다.
\end{problem}

