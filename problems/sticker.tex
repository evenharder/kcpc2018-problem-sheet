\begin{problem}{\kcpcstickertitle}
    {표준 입력}{표준 출력}
    {\kcpcstickertime\,초}{\kcpcstickermemory\,MB}{}{\kcpcstickerscore}
    
    곧 2018년이 끝나고, 2019년이 온다. 근우는 2019년에는 꼭 다이어리를 쓰기로 했다. 하지만, 처음 써보는 다이어리에 쓸 내용이 없어 고민하던 중 자신의 목표 연봉을 다이어리 앞에 쓰기로 했다.
    
    다이어리를 쓰는 사람은 알겠지만 예쁜 다이어리의 핵심은 스티커다. 그렇기 때문에 근우는 목표 연봉을 손으로 쓰지 않고, 스티커로 붙이려고 한다. 목표연봉이 100이라면 \texttt{[1][0][0]}과 같이 붙이는 것이다. \texttt{[1]}이란 1이 써져있는 스티커로, 다른 숫자에 대해서도 동일한 규칙이 적용된다.
    
    근우는 자신의 연봉 최댓값이 $ N $임을 안다. 그렇기에 근우는 0부터 $ N $까지의 수를 하나씩 스티커를 통해 모두 표현하고자 한다. 최댓값 $ N $이 10이면 만드는 과정은 다음과 같다.
    
    \begin{itemize}
        \item 스티커 더미에서 \texttt{[0]} 하나를 가져와 0을 표현하고, (\texttt{[0]}), 사용한 스티커를 스티커 더미로 되돌린다.
        \item 스티커 더미에서 \texttt{[1]} 하나를 가져와 1을 표현하고, (\texttt{[1]}), 사용한 스티커를 스티커 더미로 되돌린다.
        \item 9까지 마찬가지 방법으로 표현할 수 있다.
        \item 스티커 더미에서 \texttt{[0]} 하나와 \texttt{[1]} 하나를 가져와 10을 표현한다 (\texttt{[1][0]}). 이후 사용한 스티커 \texttt{[0]}과 \texttt{[1]}을 스티커 더미로 되돌린다.
    \end{itemize}
    
    그러므로 $ N $이 10 이면 스티커가 \texttt{[0]}부터 \texttt{[9]}까지 1개씩만 있으면 모두 표현할 수 있다.
    
    필요한 스티커를 사러 고려대 하나스퀘어 유니스토어에 도착한 근우는 고민이 생겼다. 스티커 팩에는 \texttt{[0]}부터 \texttt{[9]}까지 스티커가 한 장씩 밖에 없으면서 생각보다 너무 비싼 것이다! 그렇기에 근우는 0부터 $ N $까지 모든 수를 하나씩 표현할 수 있께 최소한의 스티커 팩만 사려고 한다. 
       
    근우는 매우 똑똑하지만, 스티커 팩 가격에 충격을 받아 계산할 수 없는 상태가 돼버렸다. 여러분이 근우의 최대 목표액 $ N $이 주어졌을 때, 근우가 필요한 최소 스티커 팩의 개수를 구해주자.
    
    \InputFile
    첫 번째 줄에 근우의 연봉 최댓값을 의미하는 정수 $ N $이 주어진다. $ (0 \leq N \leq 1,000,000,000) $
    
    \OutputFile
    첫 번째 줄에 근우가 0부터 $ N $까지 스티커로 표현하기 위해 구매해야 하는 스티커 팩의 최소 개수를 출력한다.

    \Examples
    
    \begin{example}
        \exmp{
            88
        }{%
            2
        }%
    \end{example}
     
\end{problem}

